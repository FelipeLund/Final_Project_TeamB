The goal of this project was to analyse and investigate weather data from the Swedish Meteorological and Hydrological Institute (SMHI). The datasets considered in this project contain a number of temperature measurements for each day during the past $\sim100$ years. Using the temperature data, it is possible to investigate multiple theories and research within meteorology. 
\newline
In this project, the daily temperatures on a specific day (throughout the years) is investigated and plotted in a histogram. Through the histogram it will be possible to observe the behavior of the daily temperatures, and from its distribution it could be concluded whether it has a standard Gaussian behavior. Alternatively, if the behavior and trend of the daily temperatures tend to have a more smeared out peak, or a second peak for higher temperatures, possible connections can be drawn to global warming. The project also investigates the correlation between the maximum and minimum temperatures of each year. Finally, investigation of the season length is considered. Specifically, by counting the days passed between every year's maximum temperature, it is possible to conclude whether there always are about 365 days between each midst of the summer, or if there is a certain trend of the seasonal year getting longer/shorter.


To perform a proper and convenient analysis of the weather data, the datasets from the SMHI first needed to be 'cleaned' and stored as a csv file that could be easily read by the csv file parsing library that was used. For that purpose, the two bash scripts 'ProduceCleanDatasets.sh' and 'smhicleaner.sh' were written. The smhicleaner.sh script uses bash commands to look for the city of interest in the dataset folder containing files for the different cities. If the file for the city is found, unnecessary strings and lines are removed keeping only the rawdata of the date, time, temperature, and quality (along with these as column headers). The data in the rows are changed to be separated by commas so it is now in csv format. The ProduceCleanDatasets.sh script , which should be run from the base folder 'Final\_Project\_TeamB', takes the city name as a input parameter and uses the smhicleaner.sh script to do the cleaning of the dataset and put the clean datafile with the name Clean'City' in the code/ClnData folder where it is to be accessed and used by the other C++/Root coded functions.     
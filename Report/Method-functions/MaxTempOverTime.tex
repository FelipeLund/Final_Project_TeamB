The algorithm used for the function \texttt{maxTempOverTime()} is that the CSV parser is used to retrieve a list with all the years available. Then, for each year in the list, the code stores the daily temperatures in a vector and once the year is over, it finds the maximum temperature and pushes it to another vector called \texttt{max\_temp\_vector}. This vector is then used to create a plot of the maximum temperature over time and to look for trends. The \texttt{minTempOvertime()} function works the same way except that it pushes the minimum temperature rather than the max. 
To find the daily mean temperature over time, we first parse through the file and extract every date possible and store it in a vector called \texttt{all\_datums}. For efficiencies' sake it was important that these are ordered cronologically. Then, for each date in \texttt{all\_datums}, the parser goes through the lines of the files and stores the temperature in a vector if the dates correspond to each other. Once they stop corresponding the temperatures are averaged and pushed to a vector called \texttt{dailyTempOverTime()}.
\newline
\newline
To find the days at which the max average daily temperatures occur per year, the function \texttt{Day\_MaxAvgTemp()} was written. This function parses through the dataset file for a city and for each year, it finds the daily average temperatures and stores them in a vector. A vector of the dates in the year is also stored. The maximum temperature in this vector for the current year is recorded and the index of the maximum temperature in the vector is located with another function \texttt{getIndexOfTemp} that was written. The index (i.e) reference in the date vector for the year corresponds to the day of this max temperature, and is stored in another vector \texttt{Day\_MaxAvgTemp\_Year}. As all the years are looped through the file, all the dates for the yearly maximum temperatures are stored.
\newline
\newline
Next, to find the distance (i.e days) between the yearly maximum temperatures, the dates returned by the \texttt{Day\_MaxAvgTemp()} function is compared to all the dates in the datafile, with a day counter that keeps track of the day number starting from the first day of the file  to the last. When the dates match, the day counter is stored as the last element in a \texttt{DayNoList} vector. The difference between this last element in \texttt{DayNoList} vector and the next day counter (when dates match again) is found to find the distance between the yearly maximum temperatures and stored in the \texttt{DiffList} vector. 
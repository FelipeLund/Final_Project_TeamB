In this portion of the project, a month and a day is given as input by the user (along with the filepath containing the weather data of the city in question) and the function tempOnDay parses through the CSV, extracts the temperatures for the given day, plots it as a histogram and then performs a Gaussian fit on it. The algorithm for extracting the data starts by turning the inputted month and day into a string that can be compared with the dates provided in the CSV. Then, the downloaded CSV parser is used to read the file line by line within a specific year and then pushes the temperature value for the specific day into a vector. This is done for every year available in the data meaning that the size of the final vector is that of the amount of years available. Finally, this vector is used as the data source for a histogram plotting function.
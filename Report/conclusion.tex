In conclusion, the analysis of the temperature data recorded by SMHI can give many insights into how the weather and climate conditions have been changing over the last one or two centuries.
\newline
In the first task, the histogram of the temperature of a single day shows that the distribution acts like a Gaussian. The mean  temperatures change with seasons and also the temperatures drop slightly as one goes more north. It is difficult to evaluate the exact nature for the northern most cities due lack to enough data.
\newline
In addition, regarding the maximum and minimum temperatures, one can see that there is a general trend of the temperature fluctuating over time. As a team, we wanted to investigate and inspect if the maximum temperature will rise over time, giving a insight of global warming. Looking closer to the graphs, one can see that the temperatures slightly rises over time for Falsterbo city and Umeå. However, one knows that for some years there is inadequate data, hence an absolute conclusion cannot be taken.
\newline
Lastly, the results in the third task show that the number of days in a season tend to fluctuate around 365 days. This suggests that the definition of a year lasting for 365 days, is valid and in accordance with the length of the seasons. In fact, when we experience the hottest day of the summer, we can conclude that on average we should expect to wait 365 days until we experience the warmest day of the next summer.
\newline
For further analysis, one could try to fit the oscillations that are seen as the passage of time where each oscillation indicates a season. From the fit, more conclusions about the season, the spread of temperatures, insights of global warming and climate change could be deduced. 

